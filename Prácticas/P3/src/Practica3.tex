\documentclass{article}
\usepackage[utf8]{inputenc}
\usepackage[letterpaper,top=2cm,bottom=2cm,left=3cm,right=3cm,marginparwidth=1.75cm]{geometry}
\usepackage[spanish]{babel}
\usepackage{graphicx}
\usepackage{url}
\usepackage{pdflscape}
\usepackage{hyperref}
\usepackage{enumitem}
\usepackage{amsmath}

\hypersetup{
  colorlinks=true,
  linkcolor=blue,
  filecolor=magenta,
  urlcolor=blue,
}

\setlength{\parindent}{0pt}

\title{Compiladores: Práctica 3}
\author{Roberto Ocampo Villegas}
\date{23 de Septiembre de 2024}

\begin{document}

\maketitle

\section*{Determinación de los conjuntos \( N \), \( \Sigma \) y el símbolo inicial \( S \)}

Dada la gramática \( G = (N, \Sigma, P, S) \), con las producciones P:

{\renewcommand{\labelitemi}{--}
\begin{itemize}
    \item \( \text{programa} \to \text{declaraciones sentencias} \)
    \item \( \text{declaraciones} \to \text{declaraciones declaracion} \mid \text{declaracion} \)
    \item \( \text{declaracion} \to \text{tipo lista-var ;} \)
    \item \( \text{tipo} \to \text{int} \mid \text{float} \)
    \item \( \text{lista-var} \to \text{lista-var , identificador} \mid \text{identificador} \)
    \item \( \text{sentencias} \to \text{sentencias sentencia} \mid \text{sentencia} \)
    \item \( \text{sentencia} \to \text{identificador = expresion ;} \mid \text{if ( expresion ) sentencias else sentencias} \mid \text{while ( expresion ) sentencias} \)
    \item \( \text{expresion} \to \text{expresion + expresion} \mid \text{expresion - expresion} \mid \text{expresion * expresion} \mid \text{expresion / expresion} \mid \text{identificador} \mid \text{numero} \)
    \item \( \text{expresion} \to ( \text{expresion} ) \)
\end{itemize}
}

\begin{itemize}
    \item \textbf{Conjunto de no terminales} \( N \):
    \[
    N = \{ \text{programa}, \text{declaraciones}, \text{declaracion}, \text{tipo}, \text{lista-var}, \text{sentencias}, \text{sentencia}, \text{expresion} \}
    \]

    \item \textbf{Conjunto de terminales} \( \Sigma \):
    \[
    \Sigma = \{ \text{int}, \text{float}, \text{identificador}, \text{;} , \text{,}, \text{=}, \text{if}, \text{else}, \text{while}, \text{(}, \text{)}, \text{+}, \text{-}, \text{*}, \text{/}, \text{numero} \}
    \]

    \item \textbf{Símbolo inicial} \( S \):
    \[
    S = \text{programa}
    \]
\end{itemize}

\section*{Eliminación de ambigüedad}

Podemos ver que en la gramática dada hay una ambigüedad en la producción de expresiones:\\\\

\( \text{expresion} \to \text{expresion + expresion} \mid \text{expresion - expresion} \mid \text{expresion * expresion} \mid \text{expresion / expresion} \mid \text{identificador} \mid \text{numero} \)\\\\


Esta ambigüedad ocurre debido a que no se especifica el orden de las operaciones, es decir, las operaciones de suma, resta, multiplicación y división se pueden interpretar de distintas formas.

Para resolver esta ambigüedad, reorganizamos las producciones de las expresiones para reflejar la precedencia y la asociatividad:

\subsection*{Producciones actualizadas para las operaciones aritméticas}

\begin{align*}
\text{expresion} &\rightarrow \text{expresion\_suma} \\
\text{expresion\_suma} &\rightarrow \text{expresion\_suma } + \text{ expresion\_mult} \\
                      &\;|\; \text{expresion\_suma } - \text{ expresion\_mult} \\
                      &\;|\; \text{expresion\_mult} \\
\text{expresion\_mult} &\rightarrow \text{expresion\_mult } * \text{ termino} \\
                      &\;|\; \text{expresion\_mult } / \text{ termino} \\
                      &\;|\; \text{termino} \\
\text{termino} &\rightarrow \text{identificador} \\
               &\;|\; \text{numero} \\
               &\;|\; ( \text{expresion} ) \\
\end{align*}

\subsection*{Gramática sin ambigüedad}

\begin{align*}
\text{programa} &\rightarrow \text{declaraciones sentencias} \\
\text{declaraciones} &\rightarrow \text{declaraciones declaracion} \\
                     &\;|\; \text{declaracion} \\
\text{declaracion} &\rightarrow \text{tipo lista-var ;} \\
\text{tipo} &\rightarrow \text{int} \\
            &\;|\; \text{float} \\
\text{lista-var} &\rightarrow \text{lista-var , identificador} \\
                 &\;|\; \text{identificador} \\
\text{sentencias} &\rightarrow \text{sentencias sentencia} \\
                  &\;|\; \text{sentencia} \\
\text{sentencia} &\rightarrow \text{identificador = expresion ;} \\
                 &\;|\; \text{if ( expresion ) sentencias else sentencias} \\
                 &\;|\; \text{while ( expresion ) sentencias} \\
\text{expresion} &\rightarrow \text{expresion\_suma} \\
\text{expresion\_suma} &\rightarrow \text{expresion\_suma } + \text{ expresion\_mult} \\
                      &\;|\; \text{expresion\_suma } - \text{ expresion\_mult} \\
                      &\;|\; \text{expresion\_mult} \\
\text{expresion\_mult} &\rightarrow \text{expresion\_mult } * \text{ termino} \\
                      &\;|\; \text{expresion\_mult } / \text{ termino} \\
                      &\;|\; \text{termino} \\
\text{termino} &\rightarrow \text{identificador} \\
               &\;|\; \text{numero} \\
               &\;|\; ( \text{expresion} ) \\
\end{align*}\\\\

\section*{Eliminación de recursividad izquierda}

En la gramática dada, podemos identificar casos de recursividad izquierda en las producciones de \textit{declaraciones}, \textit{lista-var}, \textit{sentencias}, \textit{expresion\_suma} y \textit{expresion\_mult}. Utilizando el algoritmo visto en clase para eliminar la recursivdad, hacemos la actualización.

\subsection*{Gramática actualizada sin recursividad izquierda}

\[
\begin{aligned}
\text{programa} &\to \text{declaraciones sentencias} \\
\text{declaraciones} &\to \text{declaracion declaraciones'} \\
\text{declaraciones'} &\to \text{declaracion declaraciones'} \;|\; \varepsilon \\
\text{declaracion} &\to \text{tipo lista-var ;} \\
\text{tipo} &\to \text{int} \;|\; \text{float} \\
\text{lista-var} &\to \text{identificador lista-var'} \\
\text{lista-var'} &\to , \text{ identificador lista-var'} \;|\; \varepsilon \\
\text{sentencias} &\to \text{sentencia sentencias'} \\
\text{sentencias'} &\to \text{sentencia sentencias'} \;|\; \varepsilon \\
\text{sentencia} &\to \text{identificador = expresion ;} \;|\; \text{if ( expresion ) sentencias else sentencias} \;|\; \text{while ( expresion ) sentencias} \\
\text{expresion} &\to \text{expresion\_suma} \\
\text{expresion\_suma} &\to \text{expresion\_mult expresion\_suma'} \\
\text{expresion\_suma'} &\to + \text{ expresion\_mult expresion\_suma'} \;|\; - \text{ expresion\_mult expresion\_suma'} \;|\; \varepsilon \\
\text{expresion\_mult} &\to \text{termino expresion\_mult'} \\
\text{expresion\_mult'} &\to * \text{ termino expresion\_mult'} \;|\; / \text{ termino expresion\_mult'} \;|\; \varepsilon \\
\text{termino} &\to \text{identificador} \;|\; \text{numero} \;|\; ( \text{expresion} )
\end{aligned}
\]

\subsection*{Proceso de factorización izquierda}

No es necesario realizar el proceso de factorización izquierda, ya que la gramática no presenta factores comunes al inicio de las producciones.


\section*{Nuevos conjuntos \( N \) y \( P \)}

Después de los procesos de eliminación de ambigüedad, recursividad izquierda y factorización izquierda, los conjuntos actualizados son los siguientes:

\subsection*{Conjunto de no terminales (\( N \))}

\[
\begin{aligned}
N = \{ &\text{programa}, \text{declaraciones}, \text{declaraciones'}, \text{declaracion}, \text{tipo}, \\
       &\text{lista-var}, \text{lista-var'}, \text{sentencias}, \text{sentencias'}, \text{sentencia}, \\
       &\text{expresion}, \text{expresion\_suma}, \text{expresion\_suma'}, \text{expresion\_mult}, \\
       &\text{expresion\_mult'}, \text{termino} \}
\end{aligned}
\]


\subsection*{Conjunto de producciones (\( P \))}

\[
P =
\begin{aligned}
& \text{programa} \to \text{declaraciones sentencias} \\
& \text{declaraciones} \to \text{declaracion declaraciones'} \\
& \text{declaraciones'} \to \text{declaracion declaraciones'} \;|\; \varepsilon \\
& \text{declaracion} \to \text{tipo lista-var ;} \\
& \text{tipo} \to \text{int} \;|\; \text{float} \\
& \text{lista-var} \to \text{identificador lista-var'} \\
& \text{lista-var'} \to , \text{ identificador lista-var'} \;|\; \varepsilon \\
& \text{sentencias} \to \text{sentencia sentencias'} \\
& \text{sentencias'} \to \text{sentencia sentencias'} \;|\; \varepsilon \\
& \text{sentencia} \to \text{identificador = expresion ;} \;|\; \text{if ( expresion ) sentencias else sentencias} \;|\; \text{while ( expresion ) sentencias} \\
& \text{expresion} \to \text{expresion\_suma} \\
& \text{expresion\_suma} \to \text{expresion\_mult expresion\_suma'} \\
& \text{expresion\_suma'} \to + \text{ expresion\_mult expresion\_suma'} \;|\; - \text{ expresion\_mult expresion\_suma'} \;|\; \varepsilon \\
& \text{expresion\_mult} \to \text{termino expresion\_mult'} \\
& \text{expresion\_mult'} \to * \text{ termino expresion\_mult'} \;|\; / \text{ termino expresion\_mult'} \;|\; \varepsilon \\
& \text{termino} \to \text{identificador} \;|\; \text{numero} \;|\; ( \text{expresion} )
\end{aligned}
\]

\end{document}
